%!TEX root = ../main.tex

\chapter{Conclusions and Future Works}
\label{chp:conclusions}

\section{Summary of Achievements}
\label{sec:achievements}

\paragraph{}
This thesis has presented a comprehensive study on vowel recognition using Convolutional Neural Networks, specifically designed for Italian language speech therapy applications. The key achievements of this research include:

\paragraph{}
First, we have successfully developed and implemented a CNN-based vowel recognition system that achieves exceptional accuracy across all Italian vowels. The model demonstrates remarkable stability and robustness, with consistent performance across different batch sizes and training configurations. Our system achieved an overall validation accuracy of 98.77%, with perfect recognition (100%) for vowels 'A' and 'U', and near-perfect accuracy exceeding 99% for other vowels. The performance remained stable across various batch sizes, demonstrating the model's robustness.

\paragraph{}
Second, the system has been successfully integrated into a web-based application platform, making it accessible and user-friendly for speech therapy applications. The implementation features real-time audio processing capabilities, interactive visual feedback mechanisms, cross-platform compatibility, and an efficient client-server architecture that ensures smooth operation across different devices and browsers.

\section{Research Implications}
\label{sec:implications}

\paragraph{}
The findings of this research have several important implications:

\paragraph{}
For Speech Therapy: The high accuracy and real-time processing capabilities of our system make it a valuable tool for speech therapists working with individuals with hearing impairments. The system's ability to provide immediate feedback can enhance the effectiveness of therapy sessions and enable more engaging practice opportunities.

\paragraph{}
For Technical Development: Our implementation demonstrates the viability of using deep learning approaches for specialized speech recognition tasks. The successful integration of CNNs with spectral analysis provides a framework for similar applications in speech processing.

\paragraph{}
For Educational Applications: The web-based nature of our solution makes it accessible for remote learning and self-practice, potentially expanding the reach of speech therapy resources.

\section{Future Research Directions}
\label{sec:future-directions}

\paragraph{}
While the current implementation shows promising results, several areas warrant further investigation:

\subsection{Data Collection and Diversity}
\label{subsec:data-future}

\paragraph{}
A primary focus for future work should be expanding and diversifying the dataset through collaboration with speech therapy clinics and schools. We propose organizing recording sessions with children of different age groups who are currently receiving speech therapy. This would provide invaluable real-world data that captures the natural variations in pronunciation patterns among children with speech disorders. The recordings should be conducted in controlled environments but with varying conditions to ensure robustness.

\paragraph{}
Additionally, we recommend collecting audio samples from different demographic groups, including various age ranges, gender distributions, and regional accents within Italy. This diversity in the training data would enhance the model's ability to generalize across different speaker characteristics and speech patterns.

\subsection{Technical Enhancements}
\label{subsec:technical-future}

\paragraph{}
Future technical developments should focus on expanding the model's capabilities to recognize more complex phonetic elements and implementing real-time noise reduction techniques. The system could be optimized for mobile devices to increase accessibility, and adaptive learning capabilities could be developed to personalize the experience for each user based on their progress and specific challenges.

\subsection{Clinical Applications}
\label{subsec:clinical-future}

\paragraph{}
Further research in clinical applications should include conducting extensive clinical trials with speech therapy patients, developing personalized training programs based on individual progress, and integrating comprehensive progress tracking and reporting features. The system could also be expanded to support other languages while maintaining its effectiveness for Italian vowel recognition.

\section{Final Remarks}
\label{sec:final-remarks}

\paragraph{}
This research has demonstrated the potential of applying modern deep learning techniques to speech therapy applications. The achieved results suggest that our approach could significantly contribute to the field of speech therapy, particularly for individuals with hearing impairments. While there are areas for improvement and expansion, the current implementation provides a solid foundation for future developments in this important domain.

\paragraph{}
The combination of high accuracy, real-time processing, and web-based accessibility makes this system a promising tool for both clinical and educational applications. As technology continues to advance, the integration of such AI-powered tools in speech therapy is likely to become increasingly important in supporting individuals with speech and hearing challenges.

