\begin{abstract}[it]

\noindent
Questa ricerca esplora l'uso delle Reti Neurali Convoluzionali (CNN) 
per riconoscere le vocali nella terapia logopedica italiana, con 
particolare attenzione al supporto di persone con disturbi del 
linguaggio.
\noindent
Lo studio affronta il bisogno di strumenti accessibili ed efficaci 
per la terapia logopedica, specialmente per bambini con problemi 
uditivi. Attraverso un sistema di riconoscimento vocale basato su 
intelligenza artificiale, il progetto mira a potenziare i metodi 
tradizionali con soluzioni tecnologiche moderne.
\noindent
Il sistema SoundRise, sviluppato presso il CSC dell'Università di 
Padova, aiuta gli utenti a migliorare la pronuncia. La piattaforma 
analizza intonazione e volume delle vocali in tempo reale. Questa 
tesi esplora come integrare l'IA nel sistema per migliorarne le 
capacità.
\noindent
La ricerca si concentra sul riconoscimento delle vocali italiane 
(/a/, /e/, /i/, /o/, /u/) usando le CNN. Il modello è addestrato 
su un ampio dataset di spettrogrammi audio, con un rigoroso processo 
di validazione per garantire precisione e affidabilità.
\noindent
I risultati mostrano l'efficacia del sistema nel classificare le 
vocali, offrendo un valido supporto per persone con disturbi del 
linguaggio. Il progetto contribuisce alla riabilitazione logopedica 
fornendo uno strumento pratico per l'apprendimento e la valutazione 
della pronuncia.

\end{abstract}