%!TEX root = ../main.tex
\begin{abstract}
\noindent
    This study addresses speech and listening impairments, highlighting the potential for improvement through targeted training. The University of Padua's Computer Engineering for Music and Multimedia (CSC) Lab is developing an online service, SoundRise, to help individuals refine their pronunciation skills. The platform provides a user-friendly interface for identifying vowel pitch and volume through audio analysis. As artificial intelligence continues to gain attention and improve various industries, this thesis investigates the feasibility of integrating AI into the SoundRise training service.
    
\noindent
    The research focuses on vowel recognition in Italian using Convolutional Neural Networks (CNNs) and introduces a new dataset covering five vowels (/a/, /e/, /i/, /o/, and /u/) for future applications. The AI technology primarily addresses the classification problem of categorizing input audio into different vowel classes. This is achieved by training the model on a substantial amount of audio data transformed into image files, with a robust validation process ensuring accuracy.
    
\noindent
    To implement the training results, this study enhances the existing SoundRise application by integrating a new SoundSpark section. This feature accepts audio input and classifies vowels directly within the interface. The results demonstrate the effectiveness of this system in recognizing vowel types, potentially enhancing outcomes for individuals with speech disorders and contributing to speech rehabilitation by offering a direct approach to learning and evaluating speech.
\end{abstract}